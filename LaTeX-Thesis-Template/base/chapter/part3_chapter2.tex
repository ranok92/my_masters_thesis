In this work, we present a data-driven IRL-based social navigation pipeline.\\
An overview of the pipeline:
\\ \textbf{An image showing the block diagram of the pipeline including the environment, the feature extractor and the other components}
We will describe each component of the pipeline in greater detail.
\section*{The IRL block:}
Inverse reinforcement learning(IRL) or inverse optimal control(IOC) has been in vogue in recent years when it comes to training robots to perform real-world tasks. This is understandable as assigning rewards to individual states to illicit out desired behaviors is challenging. IRL provides a better alternative of obtaining the underlying reward function as well as a trained agent on the obtained reward function using demonstrations from the expert.

To keep the chapter self-contained, we will briefly go over the definition of a Markov decision process, which is at the base of reinforcement and inverse reinforcement learning.
A Markov decision process or MDP can be defined as a tuple ($\mathcal{S}$,$\mathcal{A}$,T,$\gamma$, $\mathcal{R}$)  where,
\begin{itemize}
	\item $\mathcal{S}$ is the set of all possible states.
	\item $\mathcal{A}$ is the set of all possible actions.
	\item T is the state transition dynamics, i.e. the probability of moving to a state given its previous state and action, $P(s^{'}|s,a)$ .
	\item $\gamma$ is the discounting factor.
	\item $\mathcal{R}$ is the set of rewards $R:  \mathcal{S} \mapsto \mathbb{R} $ is the reward function. In practice, instead of using the raw states, hand engineered features are extracted from the states with are then used to calculate the reward of a particular state. This alleviates a lot of complexity when dealing with large continuous state-spaces. 
	\end{itemize}  
The goal of the IRL is to infer a reward function that best explains the behavior of the expert. The expert behavior is represented in terms of expert demonstrations or trajectories, $D = \{ \tau_1, \tau_2, \tau_3, \dots, 
\tau_{M} \}$ in the context of navigation. Each of these trajectories, in turn, can be further broken down into a collection of states $\tau_{i} = \{ s_{0}, s_{1}, s_{2}, \dots, s_{T} \}$ as visited by the expert in the trajectory. 

We base our work on Wulfmeier's paper, which in turn is a neural network adaptation of the work done by Ziebart 2008. 
According to the max entropy formulation, the probability of the occurrence of a trajectory is directly proportional to the reward it receives.
\begin{align}
Entropy equation - Ziebart equation 2
\end{align}
and is equal to:
\begin{align}
equation with z
\end{align}
Given a set of expert demonstrations, an optimal reward structure should maximize the probability of the occurrence of the expert demonstrations and their associated states. Mathematically, this is given by 
\begin{align}
the equation for the loglikelihood-Ziebart equation 6
\end{align}
The original paper by Ziebart used a linear combination of weights as reward functions. But linear representations have limited capabilities when it comes to expressing complex reward functions. This problem is addressed by Wulfmeier 2015 where they restructure the maxent IRL formulation using neural networks. Neural networks are universal function approximators, and this vastly improves on the amount of complexity the reward functions can encapsulate.

\begin{align}
derive the gradient expression in equation 3 of iros paper
\end{align}

The original formulation of MEDIRL was in the context of a model-based setting and the state transition matrix was used to calculate the agent SVF. While this produces an exact value of the agent SVF, assuming the availability of the state transition matrix is fairly optimistic for most real-world tasks including navigation. In an attempt to make things less constrained we take the model-free approach and focus on calculating the SVF using sampling-based methods. 
The SVF calculation:
The main challenge going model free is the calculation of the Z value, which previously could be calculated using dynamic programming [citation of the paper]. 
Under the assumption of a model-free but deterministic environment, the SVF of a deterministic policy can be reasonably computed by taking trajectory samples of the policy from the context of all the existing pedestrians in the scene. 
\begin{align}
equation 4 from iros2020
\end{align}
where the $\mathcal{P}$ represents state transitions obtained from sampling and not the state transition dynamics. \textbf{We argue that this assumption is reasonable in a navigation setting because the task is not inherently uncertain, and most transition dynamic uncertainty can be attributed to sensory noise and control error. We summarize our approach in algorithm 1}
\begin{algorithm}
	algorithm 1 from iros2020
\end{algorithm}

For solving the MDP, we employ actor-critic methods, which we will describe in detail in the next section.
\subsection*{Overview of the algorithm used}
The algorithm trains for two networks, the reward network that, given the features of a state returns the reward associated with it,\\
\textbf{equation}\\ stating this.
and the policy network, which given the same, returns the best possible action.\\
\textbf{equation}\\
The method starts with randomly initializing the weights of the reward network. This reward network is then used in the  RL block to train an agent which is optimal for the current reward structure. Once, an optimal policy is obtained, the policy is then sampled from, in the environment to obtain roll outs or trajectories in this case. A trajectory is given by the sequence of states visited by the agent {s1, s2, ... sn}.
Once the trajectories are obtained, they are used to calculate the state visitation frequency. The difference between the expert and the agent SVF is used to calculate the loss
\textbf{equation}
This loss is then back propagated through the reward network to update the weights.
Once the weights are updated, the new network is again fed into the RL block. This iterative process continues until completion.
Explanation of the L1 regularization over l2 regularization 

\section*{The RL block:}
Theory of actor-critic method. A type of policy gradient method. Policy gradient methods are a class of reinforcement learning methods where the policy is iteratively improved.


\section*{The feature extractor}:
The feature extractor is a vital component in the navigation pipeline as it acts as the medium using which the learning algorithm interacts with the environment something that provides the agent with the information/context to act upon. 
We have tested with different feature representations and have found that it plays a vital role, as the feature representations provide the information to the agent on which we want them to base their decision on.

The feature extractor can primarily be broken down into 2 broad components: local information and global information. The local information, as the name suggests, uses the information from the nearby surroundings of the agent and captures that in a binary feature vector. This provides an approximate idea of the obstacles in the vicinity. The global information provides a rough direction of where the goal of the agent is. We believe that this provides the agent with a purpose of navigation, preventing it from just rambling around the map while the local information is used to avoid obstacles and maintain proper decorum while moving towards the goal.
Although the environment publishes all the information that is there to know about the obstacles in the map, including their location, orientation and velocity, and also the exact coordinates of the goal location, we recognize that getting access to this kind of information in real life on a mobile robot navigating any given scene in the real world is extremely hard to obtain. 
This is again addressed by the feature extractor, which simultaneously acts as an information moderator, sorting information from the environment and presenting it in such a way that it can realistically be obtained by a mobile robot trying to navigate in the real world.
Both the local and the global components along with their subcomponents are described in greater detail below.
Talk about the relative orientation calculation


\subsection*{The global information}
The global information is further comprised of 3 elements:
The relative goal orientation: 
This acts as a compass, providing a rough estimate of the location of the goal based on the current position and orientation of the agent. This is denoted by a 9x1 one-hot vector, where the presence of the goal in any one of the bins is marked by a '1' keeping the rest to '0'. The 360 degrees around the agent are divided into 8 equal divisions forming the first 8 bins. The 9th bin denotes the contact of the agent with the goal. The structuring of the bins is shown below. 
The aforementioned way of getting the relative orientation concerning the agent is employed to get the relative orientation of various entities and not just the goal as we will see in subsequent sections.

The change in orientation: Represented by a 4x1 one-hot vector, the change in orientation captures the magnitude of the change in the orientation of the agent in consecutive steps. The entire range of [0-180] is divided asymmetrically into 4 divisions. The rationale behind the uneven distribution is that empirically we have found that when humans move,  they do not tend to change their orientation drastically. And having a finer resolution in the lower range help capture the nuances in this behavior in greater detail leading to better encapsulation of the essence of the navigational pattern. The division of the range is shown below.

The deviation from goal: Represented by a 4x1 one-hot vector, the deviation from goal captures the magnitude of the angle between the vector to the goal from the current position of the agent and the current orientation vector of the agent. This value can range from 0-180 which is equally divided into 4 bins as shown below.

\subsection*{The local information}
The purpose of the local information is to cram is as much information as possible in the most succinct way. Taking inspiration from previous works in this field, we have used spatial bins to effectively break the region surrounding the agent into discrete parts.

Creation of the bins:
There are two layers of spatial bins surrounding the agent and is denoted by constructing two concentric circles around the agent. The region between the agent and the inner circle is then broken into 8 equal divisions and they comprise of bin 1-8. Similarly, the region between the first and the second bin is again divided into 8 equal divisions which form bins 9-16.
Calculation of the risk:
'Risk' is a term we concoct to measure much of a  threat the agent is facing at a given time from any of these bins. Here 'threat' can loosely be seen as a measure of the possibility of hitting an obstacle. The higher the threat, the greater the chance that if the agent and the obstacles continue on their current course it will end in a collision.
The 'risk' can be seen in 3 levels:
high low and something in between.
High risk:
When the relative motion of an obstacle is towards the agent.
Low risk:
When the relative motion of an obstacle is away from the agent.
Med risk:
If not any of the above, it is classified as medium risk.
Mathematically, they are calculated as follows:

relative orient = obs orientation - agent orientation
relative distance = agent position - obs position
Let the angle between the two vectors is theta, then it is classified as high risk when
theta is less than 90 and tan theta * relative distance < threshold
where the threshold depends on the radius of the agent and that of the obstacle at hand.
if theta > 90, that means the obstacle is moving away from the agent, and hence chances of collision with the agent are highly unlikely resulting in the classification of low risk.

One thing to note is the risk is calculated for individual obstacles present in the bin separately. And it is not uncommon to have more than one obstacle falling in different risk divisions from the same spatial bin. In that case, the risk value assigned to that bin is the highest risk posed among all the obstacles that fall under that spatial bin.

\section*{The SVF calculation}:
Calculation of the agent SVF:
Calculation of the expert SVF:
Challenges of Inverse reinforcement learning in the task of social navigation:



