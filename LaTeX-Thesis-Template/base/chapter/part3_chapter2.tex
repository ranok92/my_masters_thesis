In this work, we present a data-driven IRL-based social navigation pipeline.\\
An overview of the pipeline:
\\ \textbf{An image showing the block diagram of the pipeline including the environment, the feature extractor and the other components}
We will describe each component of the pipeline in greater detail.
\section*{The IRL block:}
Inverse reinforcement learning(IRL) or inverse optimal control(IOC) has been in vogue in recent years when it comes to training robots to perform real-world tasks. This is understandable as assigning rewards to individual states to illicit out desired behaviors is challenging. IRL provides a better alternative of obtaining the underlying reward function as well as a trained agent on the obtained reward function using demonstrations from the expert.

To keep the chapter self-contained, we will briefly go over the definition of a Markov decision process, which is at the base of reinforcement and inverse reinforcement learning.
A Markov decision process or MDP can be defined as a tuple ($\mathcal{S}$,$\mathcal{A}$,T,$\gamma$, $\mathcal{R}$)  where,
\begin{itemize}
	\item $\mathcal{S}$ is the set of all possible states.
	\item $\mathcal{A}$ is the set of all possible actions.
	\item T is the state transition dynamics, i.e. the probability of moving to a state given its previous state and action, $P(s^{'}|s,a)$ .
	\item $\gamma$ is the discounting factor.
	\item $\mathcal{R}$ is the set of rewards $R:  \mathcal{S} \mapsto \mathbb{R} $ is the reward function. In practice, instead of using the raw states, hand engineered features are extracted from the states with are then used to calculate the reward of a particular state. This alleviates a lot of complexity when dealing with large continuous state-spaces. 
	\end{itemize}  
The goal of the IRL is to infer a reward function that best explains the behavior of the expert. The expert behavior is represented in terms of expert demonstrations or trajectories, $D = \{ \tau_1, \tau_2, \tau_3, \dots, 
\tau_{M} \}$ in the context of navigation. Each of these trajectories, in turn, can be further broken down into a collection of states $\tau_{i} = \{ s_{0}, s_{1}, s_{2}, \dots, s_{T} \}$ as visited by the expert in the trajectory. 

We base our work on Wulfmeier's paper, which in turn is a neural network adaptation of the work done by Ziebart 2008. 
According to the max entropy formulation, the probability of the occurrence of a trajectory is directly proportional to the reward it receives.
\begin{align}
Entropy equation - Ziebart equation 2
\end{align}
and is equal to:
\begin{align}
equation with z
\end{align}
Given a set of expert demonstrations, an optimal reward structure should maximize the probability of the occurrence of the expert demonstrations and their associated states. Mathematically, this is given by 
\begin{align}
the equation for the loglikelihood-Ziebart equation 6
\end{align}
The original paper by Ziebart used a linear combination of weights as reward functions. But linear representations have limited capabilities when it comes to expressing complex reward functions. This problem is addressed by Wulfmeier 2015 where they restructure the maxent IRL formulation using neural networks. Neural networks are universal function approximators, and this vastly improves on the amount of complexity the reward functions can encapsulate.

\begin{align}
derive the gradient expression in equation 3 of iros paper
\end{align}

The original formulation of MEDIRL was in the context of a model-based setting and the state transition matrix was used to calculate the agent SVF. While this produces an exact value of the agent SVF, assuming the availability of the state transition matrix is fairly optimistic for most real-world tasks including navigation. In an attempt to make things less constrained we take the model-free approach and focus on calculating the SVF using sampling-based methods. 
The SVF calculation:
The main challenge going model free is the calculation of the Z value, which previously could be calculated using dynamic programming [citation of the paper]. 
Under the assumption of a model-free but deterministic environment, the SVF of a deterministic policy can be reasonably computed by taking trajectory samples of the policy from the context of all the existing pedestrians in the scene. 
\begin{align}
equation 4 from iros2020
\end{align}
where the $\mathcal{P}$ represents state transitions obtained from sampling and not the state transition dynamics. \textbf{We argue that this assumption is reasonable in a navigation setting because the task is not inherently uncertain, and most transition dynamic uncertainty can be attributed to sensory noise and control error. We summarize our approach in algorithm 1. (Taken word-to-word from iros manuscript)}
\begin{algorithm}
	algorithm 1 from iros2020
\end{algorithm}

For solving the MDP, we employ actor-critic methods, which we will describe in detail in the next section.

%\subsection*{Overview of the algorithm used}
%The algorithm trains for two networks, the reward network that, given the features of a state returns the reward associated with it,\\
%\textbf{equation}\\ stating this.
%and the policy network, which given the same, returns the best possible action.\\
%\textbf{equation}\\
%The method starts with randomly initializing the weights of the reward network. This reward network is then used in the  RL block to train an agent which is optimal for the current reward structure. Once, an optimal policy is obtained, the policy is then sampled from, in the environment to obtain roll outs or trajectories in this case. A trajectory is given by the sequence of states visited by the agent {s1, s2, ... sn}.
%Once the trajectories are obtained, they are used to calculate the state visitation frequency. The difference between the expert and the agent SVF is used to calculate the loss
%\textbf{equation}
%This loss is then back propagated through the reward network to update the weights.
%Once the weights are updated, the new network is again fed into the RL block. This iterative process continues until completion.
%Explanation of the L1 regularization over l2 regularization 

\section*{The RL block:}

Actor-critic methods are a class of reinforcement learning algorithms that are built upon policy gradient methods. 
In policy-gradient methods, the goal is to iteratively improve the performance of a given policy which is achieved by maximizing the expected return of the policy. Mathematically, the objective of a policy gradient method can be expressed as:
\begin{align}
maximize \;\; J( \theta )  &\; = \; \mathbb{E} [ R | \pi_{\theta} ] \\
					   & \; = \; \mathbb{E}[ \sum^{T-1}_{t=0} r_{t+1}| \pi_{\theta}] 
\end{align}
where, $r_{t}$ is the reward obtained at time $t$ and $\pi_{\theta}$ is the policy with parameters $\theta$.\\
This leads to an update function:
\begin{align}
\delta J ( \theta ) = 
\end{align} 
Vanilla policy gradient methods suffer from high biases because they do not have a normalizing factor for the second term. This is addressed by introducing a 
baseline. 
\begin{align}
Equation with the baseline to the vanilla policy gradient.
\end{align}
The baseline can be calculated in various ways. One of the resulting algorithms is the A2C or the Advantage actor-critic algorithm, which we use to solve our MDP.
The A2C method:
Two symbiotic agents at play here. The actor and the critic. 
The critic estimates the value function of a given state.
The actor uses this information to update its policy distribution.
\begin{algorithm}
	The actor-critic algorithm.
\end{algorithm}


\section*{The feature extractor}
The feature extractor is a vital component in the navigation pipeline. It acts as a medium that enables the learning algorithm to interact with the environment. The performance of a learning agent is highly contingent on the design of the feature extractor. \cite{vasquez_et_al}\\

The feature vector generated by the feature extractor can be broken down into 2 broad components: local and global components.
\begin{itemize}
	\item The \textbf{local component}, as the name suggests, comprises of the information from nearby surroundings of the agent captured in the form of a binary feature vector. This provides an approximate idea of the obstacles in the vicinity.
	\item The \textbf{global component}, on the other hand, provides a rough direction of where the goal is with respect to the agent. This provides the agent with a purpose of navigation, preventing it from just rambling around the map.
\end{itemize}
 The environment, by design, publishes a detailed description of obstacles in the map, including their location, orientation, and velocity, and the goal. Having access to such a detailed picture of the surroundings on a mobile robot navigating any given scene in the real world is highly unlikely and difficult to obtain. This is additionally addressed by the feature extractor, which also acts as an information moderator, receiving raw information from the environment and packaging it in a feature vector that can be readily constructed by a mobile robot on the go using off-the-shelf sensors.\\
 Both the local and the global components along with their subcomponents are described in greater detail below.
Talk about the relative orientation calculation


\subsection*{The global component}
The global information is further comprised of 3 elements:

\subsubsection*{Relative goal orientation} 
This acts as a compass, providing a rough estimate of the location of the goal based on the current position and orientation of the agent. This is denoted by a $9 \times 1$ one-hot vector, where the presence of the goal in any one of the bins is marked by a $1$ keeping the rest to $0$. The $360 \degree$ around the agent are divided into $8$ equal divisions forming the first 8 bins of $45 \degree$ each. The $9^{th}$ bin denotes the contact of the agent with the goal. The structuring of the bins is shown below. 
\begin{figure}[!htbp]
	Figure showing the relative goal orientation
\end{figure}

\subsubsection*{Change in orientation}
Represented by a $5 \times 1$ one-hot vector, the change in orientation captures the magnitude of the change in the orientation of the agent in consecutive steps. The value ranges between $0 \degree$ -  $ 180 \degree$. This value is binned in one of 5 asymmetric bins. The rationale behind the uneven distribution is that empirically we have observed that human motion is smooth. So, the bins are constructed in a way to have a finer resolution in the lower range. This helps capture the nuances in the human motion in greater detail leading to better encapsulation of the essence of the navigational pattern. The division of the range is shown below.

\begin{table}[htbp]
	\caption{Bin thresholds orientation change features.}
	\label{orientation-change-bins}
	\begin{center}
		\renewcommand{\arraystretch}{1.3}
		\begin{tabular}{|c|c|}
			\hline
			Feature & Threshold \\
			\hline
			$\phi_{O1}$ & $\alpha_{OC} \in \left[ 0 , \frac{\pi}{9} \right)$ \\
			
			$\phi_{O2}$ & $\alpha_{OC} \in \left[ \frac{\pi}{9} , \frac{2\pi}{9} \right)$ \\
			
			$\phi_{O3}$ & $\alpha_{OC} \in \left[ \frac{2\pi}{9} , \frac{3\pi}{9} \right)$ \\
	
			
			$\phi_{O5}$ & $\alpha_{OC} \in \left[ \frac{3\pi}{9} , \frac{4\pi}{9} \right)$ \\
			
			$\phi_{O6}$ & $\alpha_{OC} \in \left[ \frac{4\pi}{9} , \pi \right)$ \\
			\hline
		\end{tabular}
	\end{center}
\end{table}
\subsection*{Deviation from the goal}
Represented by a $4 \times 1$ one-hot vector, the deviation from goal captures the magnitude of the angle between the vector to the goal from the current position of the agent and the current orientation vector of the agent. The value ranges from $0 \degree$ - $ 180 \degree$ which is again asymmetrically divided into 4 bins, with greater emphasis is laid on the lower degrees due to reasons mentioned earlier.

\begin{table}[htbp]
	\caption{Bin thresholds for deviation from the goal.}
	\label{deviation-from-goal-bins}
	\begin{center}
		\renewcommand{\arraystretch}{1.3}
		\begin{tabular}{|c|c|}
			\hline
			Feature & Threshold \\
			\hline
			$\phi_{GA1}$ & $\alpha_{GA} \in \left[ 0 , \frac{\pi}{8} \right)$ \\
			
			$\phi_{GA2}$ & $\alpha_{GA} \in \left[ \frac{\pi}{8} , \frac{\pi}{4} \right)$ \\
			
			$\phi_{GA3}$ & $\alpha_{GA} \in \left[ \frac{\pi}{4} , \frac{3\pi}{4} \right)$ \\
			
			$\phi_{GA4}$ & $\alpha_{GA} \in \left[ \frac{3\pi}{4} , \pi \right]$ \\
			\hline
		\end{tabular}
	\end{center}
\end{table}

\vspace{3cm}
\subsection*{The local information}
Taking inspiration from previous works in this field \cite{fahad-et-al} \cite{vasquez-et-al}, we use spatial bins to effectively break the region surrounding the agent into discrete segments and calculate a 'risk' metric for each of these bins.

\subsubsection*{Creation of the bins}
There are 16 spatial bins surrounding the agent. They are arranged in two concentric circles around the agent. The region between the agent and the inner circle is then broken into 8 equal divisions. These form bins from 1 - 8. Similarly, the region between the first and the second bin is again divided into 8 equal divisions which form bins 9 - 16.
\subsubsection*{Calculation of the risk}
'Risk' is a term we label the 'threat' the agent faces at a given time from any of these bins. Here 'threat' can loosely be seen as a measure of the possibility of hitting an obstacle. The higher the threat, the greater the chance that if the agent and the obstacles continue on their current course it will end in a collision.
The 'risk factor' is divided in 3 levels: high, low and something in between.
\begin{itemize}
	\item High risk:
When the relative motion of an obstacle is towards the agent and can lead to a collision if not intervened.
	\item Low risk:
When the relative motion of an obstacle is away from the agent.
	\item Med risk:
Anything in between
\end{itemize}
The calculation of the risk values are based on the following entities:
\begin{align}
	\vec{o}_{rel} = & \;\; \vec{o}_{obs} - \vec{o}_{agent}  \\
	\vec{d}_{rel} =  &\;\; \vec{d}_{agent} - \vec{d}_{obs} \\
	\theta_{risk} =  & \;\; \angle (\vec{o}_{rel}, \vec{o}_{rel}) \\
	\mathbf{s}_{obs} = & \;\; \tan(\theta_{risk}) \times |\vec{d_{rel}}| \\
	\mathbf{T} = & \;\; \text{agent witdh} + \text{obstacle witdh}
\end{align}
where $\vec{o}_{rel}$ is the relative orientation of the obstacle w.r.t the agent, $\vec{d}_{rel}$ is the relative position of the agent w.r.t to the obstacle, $\theta_{risk}$ is the angle between the vectors, $\vec{o}_{rel}$ and $\vec{d}_{rel}$,  $\mathbf{s}_{obs}$ is the estimated \textbf{safety margin} between the agent and the obstacle and $\mathbf{T}$ is a predefined value which if maintained between the agent and an obstacle should guarantee a collision-free trajectory.\\
An obstacle is marked as 'high risk' when $\theta$ is less than $90\degree$ and the safety margin is less than $\mathbf{T}$. If $\theta$ < $90 \degree$, this indicates that the obstacle is moving away from the agent and hence chances of collision are less and hence low risk. Anything that does not fall in the above two categories are considered as medium risk. The risk calculation conditions and values are summarized in Table \ref{risk-threshold-table}

\begin{table}[htbp]
	\caption{Thresholds for calculating risk.}
	\label{risk-threshold-table}
	\begin{center}
		\renewcommand{\arraystretch}{1.3}
		\begin{tabular}{|c|c|}
			\hline
			Risk value & Risk condition \\
			\hline
			High & $\theta < 90\degree$ \&  $\mathbf{s}_{obs}$ < $\mathbf{T}$   \\
			
			Low & $\theta > 90\degree$\\
			
			Medium & otherwise \\
			\hline
		\end{tabular}
	\end{center}
\end{table}
The risk value of each bin is represented using a $3 \times 1$ one-hot vector. One thing to note is the risk is calculated for individual obstacles present in the bin separately. And it is not uncommon to have more than one obstacle falling in different risk divisions from the same spatial bin. In that case, the risk value assigned to that bin is the highest risk posed among all the obstacles that fall under that spatial bin.
\begin{figure}[!htbp]
	\text{Graphics that shows each of the risk conditions}
\end{figure}

\section*{The SVF calculation}
One of the main challenges of IRL is calculating the expected state distribution of a given RL trained agent. (equation x). In model-based environments, this can be calculated using dynamic programming \cite{wulfmeier-deepirl}. We find the assumption of the access to the state transition matrix for a real world setting a bit too ambitious and instead assume a model-free but deterministic environment. \textbf{Explanation as to why this is a reasonable assumption} This is a reasonable assumption in the context of a navigation problem because, the movements of pedestrians in general are inherently deterministic and any observed uncertainty by the agent can be attributed to the error in measurement by the onboard sensors.\\

The lack of state dynamics forces us to look into sampling techniques to get a reasonable estimate of the svf. While this can be time consuming and computationally expensive, the problem is drastically simplified with greedy policies. (greedy soft actor critic).\textbf{Why?} 
Svf calculation for an agent following a greedy policy in a deterministic environment can be calculated by taking a single sample trajectory for each pedestrian by replacing them with the agent and letting it run till completion. The calculation of the expected state visitation frequency of the agent is shown in equation \ref{agent-approx}
\begin{align}
\label{agent-approx}
\text{Equation to calculate the state visitation frequency}
\end{align}
Another thing we experiment with is smoothing of the svf. \\
\textbf{Why smoothing?}\\
To explicitly take into consideration the uncertainty that might arise due to measurements from on-board sensors. The goal : instead of assigning increasing the SVF value for a given state by $1$ on observation, get a probability distribution of all the nearby states that might be occurring given the observed state and use that distribution to update the SVF instead.\\
\textbf{Equations to describe the smoothing process.}
As the state vector is composed of different components, the probabilistic distribution is applied to each of the components separately.
\begin{itemize}
	\item Smoothing of the goal.
	\item Smoothing of the deviation from goal.
	\item Smoothing of the change in orientation.
	\item Smoothing of the risk information.
	\item Smoothing of the speed information.
\end{itemize}


