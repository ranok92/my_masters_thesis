\part{\rm\bfseries Introduction}
\label{ch:introduction}
\begin{enumerate}
    \item \textbf{What is the problem and why even bother with this.}\\
    The problem is two pronged. 
    \begin{enumerate}
        \item How to make robots navigate.
    The field of autonomous navigation has come a long way, starting from simple methods like potential field, which in itself produces quite remarkable results to the recent use of deep-learning.
    \end{enumerate}
    Today's society is going head first towards a future with autonomous machines. Even today we have al shit ton of autonomous machines that does a lot of things for us, starting from the alarm clock in the morning to the self checkout machines, there are a lot going on. But the main underlying theme here is autonomy. Although they work fine, a vital property that is missing from these machines is locomotion. Imagine how hindered we would be if we could not move about at our own will. It would highly debilitate our day to day activity. The same goes for robots, a moving robot is a more useful robot, which is why a lot of research in the past couple of decades has gone in the problem of navigation.
    \item \textbf{What part of it has been overcome} This will include some of the current and past works that has gone to address the problem.\\
    The field of autonomous navigation has come a long way, starting from simple methods like potential field, which in itself produces quite remarkable results to the recent use of deep-learning.
    \item \textbf{What problem still remains and why is that worth exploring.}
    \item \textbf{A brief introduction of the works that are addressing these problems and how they perform}
    \item \textbf{What are you trying to convey in this thesis} The aim of your research.
    \item \textbf{Outline the order of information in the thesis}
    \item \textbf{Outline the methodology}
\end{enumerate}

\textbf{The motivation for the problem of social navigation.}
Stationary robots are okay, but robots that move about? That is the real deal, which is why autonomous navigation has been a hot topic for the past couple of decades. The task of navigation can be seen as a two-stage problem involving:
\begin{enumerate}
    \item A high-level planning problem.
    \item A lower-level control problem.
\end{enumerate}
Given a start and a final goal, the high-level planning is responsible for chalking out a near-optimal feasible path (where optimality depends on the task at hand, be it the shortest path, or the path with the least congestion or maybe get a drink on the way, etc.) Once the high-level planner produces a general plan of navigation (maybe return a set of control points describing the path) the responsibility of the lower-level controller is to go from one point to the next in the best possible way.
Here the term 'best' is a loose term with hardly a solid definition. Different metrics can be used to score a controller. Some of the obvious comparison points will be:
\begin{enumerate}
	\item Was the controller able to reach the control point as planned out by the high-level planner.
	\item How much time it took for the local controller to navigate the segment.
	\item Did it hit something that it was not supposed to?
\end{enumerate}
But after that, things start becoming a bit blurry, and it becomes difficult to pinpoint characteristics of the controller as to what makes the controller better than another. It is not a one-size-fits-all kind of a strategy. This lack of an ironclad rule-set becomes more apparent when the goal is to navigate among humans.
Few other talking points:
\begin{enumerate}
	\item Existing works and their performance and issues in one line or so.
	\item Freezing robot problem.
	\item Any kind of study that is involved with finding how to rate a controller/path. 
\end{enumerate}


\subsubsection*{What is the problem and why is it worth exploring:}
    The problem at hand is navigation in robots. Basically, this means we want the robot to go from point A to point B and in doing so, answer the two questions: was it able to reach the goal? if yes, was the way it reached the goal acceptable. Now, acceptable is somewhat a relative term and more on this later. But there are a few characteristics in a trajectory which is inevitably desirable no matter what like, the absence of hard/damaging collisions in the trajectory, or taking a reasonable amount of time to complete a trajectory. 
    
    It is worth exploring because, nowadays, robots are getting more and more capable, and can perform various tasks with ease and, if they could also move, then that would be great because that would make our life so much easier. Most of the real life tasks need some kind of locomotion and with autonomy in locomotion a robot can become that much more impactful.

\subsubsection*{What work has already been done if any. How does it addresses the above problem. What do they miss out on?}
    For the past couple of decades, there is a lot of work going on this field. Starting from the work by XYZ on potential fields there has been a ton of work addressing this problem, each iteratively improving on the previous. 
