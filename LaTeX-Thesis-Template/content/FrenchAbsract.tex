%R¨¦sum¨¦

Un mod\`ele est une abstraction d¡¯un syst\`eme r\'eel. Pour la conception de syst\`emes complexes, la mod\'elisation est pr\'ef\'er\'ee aux m\'ethodes traditionnelles, parce que la mod\'elisation permet d¡¯analyser et de simuler avant la mise en {\oe}uvre. De plus, les outils de g\'en\'eration de code fournis par la mod¨¦lisation aident \`a produire des codes sans d\'efaut. La mod\'elisation dans des domaines sp\'ecifiques et des m\'eta-mod\`eles (l'abstraction des mod\`eles) fournissent des syntaxes et des environnements sp\'ecifiques aux mod\'elisateurs. Le m\'eta-m\'etamod\`ele (l'abstraction de m\'etamod\`eles) normalise la description des m\'etamod\`eles. Les architectures de m\'etamod\'elisation fournissent les directives \`a suivre afin d¡¯organiser les mod\`eles et les m\'etamod\`eles. Jusqu'\`a ce jour, de nombreuses normes et d¡¯outils en m\'etamod\'elisation  ont \'et\'e d\'evelopp\'es.



Cependant, deux inconv\'enients demeurent et pr\'eviennent la cr\'eation d¡¯un outil de m\'etamod\'elisation bien d\'efini. Le premier est que les architectures actuelles sont lin\'eaires. Ceci qui ne les permettent pas de diff\'erencier ad\'equatement l'aspect des diff\'erents r\^oles dans la m\'etamod\'elisation. Le second est qu¡¯elles n¡¯ont pas de caract\`ere ex\'ecutable \`a partir de la racine de la m\'etamod\'elisation, puisque la plupart des m\'eta-m\'eamod\`eles sont con{\c c}us pour d\'ecrire des informations structurelles plut\^ot que comportementaux.


Dans ce projet, nous avons utilis\'e une architecture \`a deux dimensions avec la classification logique et physique, s\'eparant ainsi le point de vue des mod\'elisateurs et celui des d\'eveloppeurs d'outils. Nous avons con{\c c}u ArkM3, un m\'eta-m\'etamod\`ele universel, auto-descriptible et ex\'ecutable. Il inclut \'egalement un langage d'action, ce qui le rend ex\'ecutable. En combinant cette architecture et ce m\'eta-m\'etamod\`ele rend possible de mettre sur pied un outil universel d¡¯amor{\c c}age de m\'etamod\'elisation. Pour d\'emontrer notre conception, nous avons construit le noyau de AToMPM (A Tool for Multi-Paradigm Metamodelling), une version mise \`a jour de AToM$^3$ (A Tool for Multi-Formalism and MetaModelling). Nous pr\'esentons \'egalement une \'etude de cas selon un syst\`eme de Petri Net ``Readers/Writers''.




%
%Abstract
%A model is an abstraction of the real system. To design complex systems, modelling is preferred to the traditional methods for its capability to analyse and simulate before implementation, and its tools for code generation which allows for defect-free code. The domain specific modelling and metamodels (the abstraction of models) provide the modellers domain specific syntax and environments. The meta-metamodel (the abstraction of metamodels) defines a unified description of various domain metamodels. Metamodelling architectures provide the guideline of organizing models and metamodels. Many metamodelling standards and tools have been developed.
%However, two drawbacks have prevented us from having a well-defined metamodelling tool. The first is that current linear architectures fail to appropriately separate the views of different roles in metamodelling. The second is the missing executability at the root of metamodelling since most meta-metamodels are designed to describe structural information rather than behavioural.
%In this project, we used a two-dimensional metamodelling architecture with logical and physical classifications that separates the view of modellers and that of tool developers. We design a general-purpose, self-describable, executable meta-metamodel ArkM3 which includes an action language and thereby enables executability. With this architecture and this meta-metamodel, we make a general-purpose, comprehensive, bootstrapping metamodelling tool possible. To demonstrate our design, we built the kernel of AToMPM (A Tool for Multi-Paradigm Metamodelling), an updated version of AToM3 (A Tool for Multi-Formalism and MetaModelling). We also present a case study that models a Readers/Writers system Petri Net model.