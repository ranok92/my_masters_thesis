\chapter*{\rm\bfseries Abr\'eg\'e}
Dans ce m\'emoire, nous pr\'esentons une m\'ethode de navigation utilisant l'apprentissage par renforcement inverse bas\'e sur l'entropie maximale (maximum entropy deep inverse reinforcement learning, MEDIRL) qui prend avantage de d\'emonstrations d'un expert pour apprendre la conformit\'e sociale.\\
La capacit\'e de naviguer de mani\`ere autonome est un pilier de la robotique mobile. Avec l'augmentation des d\'eploiements de robots \`a proximit\'e des humains, un syst\`eme de navigation respectant la conformit\'e sociale est une n\'ecessit\'e. Nous d\'efinissons la navigation socialement conforme comme m\'ethode de navigation qui, en plus de satisfaire les principes de la navigation classique, ressemble \`a la mani\`ere dont les humains se d\'eplacent dans une foule. G\'en\'eralement, cela est un amalgame de plusieurs facteurs, incluant les pr\'ef\'erences personnelles, les normes socioculturelles et les circonstances du moment ce qui rend le probl\`eme difficile.\\
La principale contribution de ce m\'emoire est l'adaptation du MEDIRL dans un environnement sans mod\`ele et un ensemble de repr\'esentation de caract\'eristiques (features), simples et pourtant originales, qui capturent l'information pertinente sur l'environnement, permettant \`a l'agent de naviguer de mani\`ere semblable \`a un humain. Nous entrainons notre m\'ethode en utilisant un ensemble de donn\'ees de quatre minutes disponible publiquement, qui contient un enregistrement de gens marchant dans un campus universitaire, et \'evaluons sa performance en la comparant \`a des m\'ethodes existantes. Nous effectuons deux exp\'eriences : premi\`erement, nous d\'emontrons la performance de l'apprentissage par renforcement inverse compar\'ee \`a d'autres m\'ethodes disponibles et deuxi\`emement, nous comparons notre repr\'esentation de caract\'eristiques \`a d'autres \'etant propos\'ees dans la litt\'erature. Dans cette \'evaluation, nous concluons que les trajectoires produites par notre approche d\'emontrent une ressemblance significative aux d\'emonstrations humaines, en plus de maintenir une performance comparable pour atteindre un but sans faire une collision sur son chemin. De plus, compar\'ee \`a d'autres repr\'esentations de caract\'eristiques, notre approche a d\'emontr\'e un taux de succ\`es significativement sup\'erieur quant \`a l'atteinte du but, avec des am\'eliorations sur l'imitation de l'expert.